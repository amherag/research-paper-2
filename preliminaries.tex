\section{Preliminaries}
\label{section:preliminaries}

\subsection{Fuzzy Systems}
\label{subsection:fuzzy-systems}

In traditional logic one can generate logical inferences, such as
\textit{if it's raining, then there are clouds in the sky}. In a
similar fashion, we can use fuzzy sets to represent the antecedents
and consequents in a logical inference process
\cite{kruse1994foundations}. For example, one can extend the previous
example to: \textit{if it's raining a lot, then there are many clouds
in the sky}.

There is a number of ways in which one can construct a fuzzy inference
system, where one or more inputs or antecedents can be used to
generate one or more outputs or consequents. Arguably, the two most
popular types of fuzzy inference systems are the ones proposed by
Mamdani and Assilian~\cite{Mamdani1975}, and Takagi and
Sugeno~\cite{Takagi1985}. These systems use a series of fuzzy sets to
represent the relationship between an input and its grade of
membership to a set. These sets usually represent adjectives that
describe the inputs, and are also considered to be the antecedents in
the fuzzy inference system. For example, an input of 0.8 can represent
a ``very high'' value. After obtaining these grades of membership, one
can use these values to ``fire'' or ``activate'' the consequents. In
the case of a Mamdani system, the consequents are represented as fuzzy
sets, just like the antecedents. In contrast, in a Sugeno system,
consequents are represented by mathematical functions. A set of rules
is used to determine the relationship between the antecedents and the
consequents, for example: \textit{if food quality is high then tip is
high}. The aforementioned rule is creating a relationship between the
fuzzy set that represents ``high food quality'' in the antecedents,
and the fuzzy set that represents ``high tip'' in the
consequents. Further continuing with the example, if ``food quality''
is represented by a value of 0.8, the rule that creates the
relationship between ``food quality'' and ``tip'' could determine a
``tip'' of 0.8 too, depending on what membership function and what
parameters are decided to be used to represent each.

We have explained how a relationship between antecedents and
consequents can be constructed in a fuzzy inference
system. Nevertheless, the most interesting problem arises when a
problem involves several fuzzy sets to represent different adjectives
for single antecedents or consequents. In these cases, depending on
the fuzzy rules, a number of consequents can be fired according to the
inputs to the system. As seen in Figure \ref{figure:antecedents}, the
input---represented by the dotted vertical black line---is associated
with three fuzzy triangular sets or antecedents, where it
``activates'' two of them. According to a set of fuzzy rules, the
inputs then fire a set of triangular fuzzy sets that represent the
consequents, as seen in Figure \ref{figure:consequents}.

The fuzzy sets that represent the consequents are cut, and new shapes
are obtained using those cuts, as represented by the green shapes in
Figure \ref{figure:consequents}. These shapes are aggregated and
result in the output of the fuzzy inference system, and this result
can then be defuzzified using different methods, such as obtaining the
centroid of the shape. In this example, a Mamdani fuzzy inference
system is considered; in the case of a Sugeno system, for example, the
antecedents would be represented by arbitrary mathematical functions,
instead of membership functions representing shapes such as the
triangles in the example presented above.

\Figure[](topskip=0pt, botskip=0pt, midskip=0pt)
[width=0.6\linewidth]
{img/antecedents.png}
{Example of antecedents in a Mamdani fuzzy system.
  \label{figure:antecedents}}

\Figure[](topskip=0pt, botskip=0pt, midskip=0pt)
[width=0.6\linewidth]
{img/consequents.png}
{Example of consequents in a Mamdani fuzzy system.
  \label{figure:consequents}}