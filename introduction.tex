\section{Introduction}
\label{section:introduction}

One of the challenges in multi-agent systems is the implementation of
a coordination control mechanism that manages the agents in a model to
collaboratively reach their goals. This challenge can be dissected
into multiple sub-challenges, which include consensus and
synchronization among the agents \cite{dorri2018multi}. Reaching a
consensus among the agents is crucial for obtaining outputs from a
model which takes into consideration the state of all of its agents,
while a synchronization mechanism allows heterogeneous agents to take
action when it is appropriate. The method proposed in this paper
addresses both challenges of consensus and synchronization in
cooperative fuzzy agent-based models. We successfully used an initial
prototype of this method in \cite{hernandez2021using}, where agents in
a multi-agent system use Mamdani intuitionistic fuzzy systems to
represent knowledge generated from Forex market datasets. The agents
fuzzy systems were used as part of what we called ``specialization
functions'', which served the multi-agent system as a synchronization
system. This paper presents an improved version of that method, which
has been extended to also function as a consensus mechanism.

In our method, agents are designed to have their rules represented as
fuzzy systems. The grades of membership calculated for the antecedents
of these fuzzy systems are used to determine a score for the agent's
competence in its current environment. %Continue

We highlight the main contributions of this work as follows:
\begin{itemize}
  \item First, agents in the agent-based models can use fuzzy systems
as both their rules and as a coordination control system, specifically
for consensus and synchronization.
%% Add rest.
\end{itemize}

%% Update for this paper. Rephrase.
The reader will find an in-depth explanation of our method in Section
\ref{section:proposed-method}, and explanations for the concepts
required to understand the method can be found in Section
\ref{section:preliminaries}. In order to evaluate the perforance of
our method, experiments were performed and are described in Section
\ref{section:experiments}. The results of the experiments are
presented in Section \ref{section:results}, and a discussion of these
results can be found in Section \ref{section:conclusion}. Finally,
Section \ref{section:future-work} discusses some directions that our
presented method can take in the future to better demonstrate its
capabilities.