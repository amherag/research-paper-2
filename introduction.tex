\section{Introduction}
\label{section:introduction}

One of the challenges in multi-agent systems is the implementation of
a coordination control mechanism that manages the agents in a model to
collaboratively reach their goals. This challenge can be dissected
into multiple sub-challenges, which include consensus and
synchronization among the agents \cite{dorri2018multi}. Reaching a
consensus among the agents is crucial for obtaining outputs from a
model which takes into consideration the state of all of its agents,
while a synchronization mechanism allows heterogeneous agents to take
action when it is appropriate. The method proposed in this paper
addresses both challenges of consensus and synchronization in
cooperative fuzzy agent-based models. We successfully used an initial
prototype of this method in \cite{hernandez2021using}, where agents in
a multi-agent system use Mamdani intuitionistic fuzzy systems to
represent knowledge generated from Forex market datasets. The agents
fuzzy systems were used as part of what we called ``specialization
functions'', which served the multi-agent system as a synchronization
system. This paper presents an improved version of that method, which
has been extended to also function as a consensus mechanism.

In our method, agents are designed to have their rules represented as
fuzzy systems. The grades of membership calculated for the antecedents
of these fuzzy systems are used to determine a score for the agent's
competence in its current environment. For this statement to make
sense, the cores ($\mu (x) = 1$) of the membership functions must
represent ideal values for the agent's inputs. As a consequence of
this design, agents can be selected among an agent pool according to
their level of competence, regardless of the agents' heterogeneity.

Agent models can achieve synchronization using our method by selecting
the most competent agent or agents to perform a task at a given
moment. A restriction that must take place is that the agent models
must be cooperative, i.e., every agent in the model must have the same
goal. On the other hand, consensus is achieved by taking a subset ($A
\subseteq B$) of those agents with the highest scores or levels of
competence, and apply an aggregation function on their outputs, e.g.,
calculating the mean of their outputs.

We highlight the main contributions of this work as follows:
\begin{itemize}
\item First, agents in the agent-based models can use fuzzy systems as
both their rules and as a coordination control system, specifically
for consensus and synchronization.
\item Second, synchronization is achieved despite the level of
heterogeneity in the agents in the agent-based model.
\item Third, consensus among the agents is obtained by using a subset
of the agents with the highest levels of competence and applying an
aggregation function on their outputs.
\item Lastly, the generated agent-based models must be cooperative.
\end{itemize}

We start this paper by discussing the state of the art regarding
coordination mechanisms for multi-agent systems and agent-based models
in \ref{section:state-of-the-art}. A detailed explanation of our
method is found in Section \ref{section:proposed-method}, and
explanations for the concepts required to understand the method can be
found in Section \ref{section:preliminaries}. Models generated by our
method are described by using the Overview, Design Concepts, and
Details (ODD) protocol \cite{Grimm2020}, and they're described in
Section \ref{section:model}. The design of the experiments that we
performed is detailed in Section \ref{section:experiments}, and the
results that demonstrate the qualitative aspects of our method are
presented in Section \ref{section:results}. We conclude the
presentation of our work by providing a conclusion and a discussion of
future work involving our method in Sections \ref{section:conclusion}
and \ref{section:future-work}, respectively.