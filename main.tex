\documentclass{ieeeaccess}
\usepackage{cite}
\usepackage{amsmath,amssymb,amsfonts}
\usepackage{algorithm}
\usepackage{algorithmicx}
\usepackage{algpseudocode}
%% \usepackage{caption}
\usepackage{graphicx}
\usepackage{textcomp}

\usepackage{bm}

\usepackage{multirow}

% \usepackage{hyperref}

\usepackage{array}
\setlength\extrarowheight{3pt}

\algnewcommand\algorithmicforeach{\textbf{for each}}
\algdef{S}[FOR]{ForEach}[1]{\algorithmicforeach\ #1\ \algorithmicdo}

\providecommand{\e}[1]{\ensuremath{\times 10^{#1}}}

\def\BibTeX{{\rm B\kern-.05em{\sc i\kern-.025em b}\kern-.08em
    T\kern-.1667em\lower.7ex\hbox{E}\kern-.125emX}}

\begin{document}
\history{Date of publication xxxx 00, 0000, date of current version xxxx 00, 0000.}
\doi{10.1109/ACCESS.2017.DOI}

\title{Fuzzy Grade of Membership as a Consensus and Synchronization Mechanism in Agent-Based Models}
\author{
    \uppercase{Amaury Hernandez-Aguila\authorrefmark{1},
    Mario Garc\'{i}a-Valdez\authorrefmark{1},
    Juan-Juli\'{A}n Merelo-Guerv\'{o}s\authorrefmark{2},
    Manuel Casta\~{n}\'{o}n-Puga\authorrefmark{3} and
    Oscar Castillo L\'{o}pez\authorrefmark{1}}}
\address[1]{National Technological Institute of Mexico, Calzada Del Tecnol\'{o}gico s/n, Fraccionamiento Tomas Aquino, Tijuana, BC 22414 Mexico (e-mail: {amerhag,mario,ocastillo}@tectijuana.edu.mx)}
\address[2]{University of Granada, Campus Aynadamar Daniel Saucedo Aranda s/n, Granada 18071, 80523 Spain (e-mail: jmerelo@geneura.ugr.es)}
\address[3]{Autonomous University of Baja California, Calzada Universidad 14418, Tijuana, BC, 22390, Mexico (e-mail: puga@uabc.edu.mx)}

\tfootnote{}

\markboth
{Author \headeretal: Preparation of Papers for IEEE TRANSACTIONS and JOURNALS}
{Author \headeretal: Preparation of Papers for IEEE TRANSACTIONS and JOURNALS}

\corresp{Corresponding author: Mario Garc\'{i}a-Valdez (e-mail: mario@tectijuana.edu.mx).}

\begin{abstract}
  The method presented in this paper serves as a coordination control
mechanism for cooperative fuzzy agent-based models. Specifically,
fuzzy systems are used to model the agents' rules. Our method focuses
on solving the challenges of synchronization and consensus among
agents. The former is achieved by calculating a competence score for
every agent by using the grades of membership for the antecedents in
each agent's fuzzy system. The latter is obtained by utilizing a
competence-ordered subset of agents from an agent pool, and applying
an aggregation function to the outputs of this subset. Synchronization
is achieved despite the agents' level of heterogeneity, but the
agent-based models generated by the method must be cooperative,
i.e. every agent must have the same goal. A series of experiments are
presented, which have the objective of demonstrating the qualitative
aspects of our method.
\end{abstract}
\begin{keywords}
% Check keywords from here:
% http://www.ieee.org/organizations/pubs/ani_prod/keywrd98.txt
  Economic forecasting, fuzzy systems, multi-agent system, activation function, forex market.
\end{keywords}

\titlepgskip=-15pt

\maketitle

\section{Introduction}
\label{section:introduction}

One of the challenges in multi-agent systems is the implementation of
a coordination control mechanism that manages the agents in a model to
collaboratively reach their goals. This challenge can be dissected
into multiple sub-challenges, which include consensus and
synchronization among the agents \cite{dorri2018multi}. Reaching a
consensus among the agents is crucial for obtaining outputs from a
model which take into consideration the state of all of its agents,
while a synchronization mechanism allows heterogeneous agents to take
action when it is appropriate. The method proposed in this paper
addresses


\input{state-of-the-art}
\section{Preliminaries}
\label{section:preliminaries}

\subsection{Fuzzy Systems}
\label{subsection:fuzzy-systems}

In traditional logic one can generate logical inferences, such as
\textit{if it's raining, then there are clouds in the sky}. In a
similar fashion, we can use fuzzy sets to represent the antecedents
and consequents in a logical inference process
\cite{kruse1994foundations}. For example, one can extend the previous
example to: \textit{if it's raining a lot, then there are many clouds
in the sky}.

There is a number of ways in which one can construct a fuzzy inference
system, where one or more inputs or antecedents can be used to
generate one or more outputs or consequents. Arguably, the two most
popular types of fuzzy inference systems are the ones proposed by
Mamdani and Assilian~\cite{Mamdani1975}, and Takagi and
Sugeno~\cite{Takagi1985}. These systems use a series of fuzzy sets to
represent the relationship between an input and its grade of
membership to a set. These sets usually represent adjectives that
describe the inputs, and are also considered to be the antecedents in
the fuzzy inference system. For example, an input of 0.8 can represent
a ``very high'' value. After obtaining these grades of membership, one
can use these values to ``fire'' or ``activate'' the consequents. In
the case of a Mamdani system, the consequents are represented as fuzzy
sets, just like the antecedents. In contrast, in a Sugeno system,
consequents are represented by mathematical functions. A set of rules
is used to determine the relationship between the antecedents and the
consequents, for example: \textit{if food quality is high then tip is
high}. The aforementioned rule is creating a relationship between the
fuzzy set that represents ``high food quality'' in the antecedents,
and the fuzzy set that represents ``high tip'' in the
consequents. Further continuing with the example, if ``food quality''
is represented by a value of 0.8, the rule that creates the
relationship between ``food quality'' and ``tip'' could determine a
``tip'' of 0.8 too, depending on what membership function and what
parameters are decided to be used to represent each.

We have explained how a relationship between antecedents and
consequents can be constructed in a fuzzy inference
system. Nevertheless, the most interesting problem arises when a
problem involves several fuzzy sets to represent different adjectives
for single antecedents or consequents. In these cases, depending on
the fuzzy rules, a number of consequents can be fired according to the
inputs to the system. As seen in Figure \ref{figure:antecedents}, the
input---represented by the dotted vertical black line---is associated
with three fuzzy triangular sets or antecedents, where it
``activates'' two of them. According to a set of fuzzy rules, the
inputs then fire a set of triangular fuzzy sets that represent the
consequents, as seen in Figure \ref{figure:consequents}.

The fuzzy sets that represent the consequents are cut, and new shapes
are obtained using those cuts, as represented by the green shapes in
Figure \ref{figure:consequents}. These shapes are aggregated and
result in the output of the fuzzy inference system, and this result
can then be defuzzified using different methods, such as obtaining the
centroid of the shape. In this example, a Mamdani fuzzy inference
system is considered; in the case of a Sugeno system, for example, the
antecedents would be represented by arbitrary mathematical functions,
instead of membership functions representing shapes such as the
triangles in the example presented above.

\Figure[](topskip=0pt, botskip=0pt, midskip=0pt)
[width=0.6\linewidth]
{img/antecedents.png}
{Example of antecedents in a Mamdani fuzzy system.
  \label{figure:antecedents}}

\Figure[](topskip=0pt, botskip=0pt, midskip=0pt)
[width=0.6\linewidth]
{img/consequents.png}
{Example of consequents in a Mamdani fuzzy system.
  \label{figure:consequents}}
\section{Proposed Method}
\label{section:proposed-method}
\section{Model}
\label{section:model}
\input{experiments}
\input{results}
\input{conclusion}
\input{future-work}

\section*{Acknowledgment}
This paper has been supported in part by project DeepBio (TIN2017-85727-C4-2-P).

\begin{IEEEbiography}[{\includegraphics[width=1in,height=1.25in,clip,keepaspectratio]{img/amaury-1by1half-in.png}}]{Amaury
    Hernandez-Aguila}

  Amaury Hernandez-Aguila received the Ph.D degree in Computer Science and the 
  M.Sc. degree in Computer Science from the Tijuana Institute of Technology,
  Mexico, in 2014 and 2019, respectively. He is currently participating in a
  post-doctoral program in Tijuana Institute of Technology, researching how
  multi-agent systems and fuzzy logic can be used for the prediction of
  financial markets.

\end{IEEEbiography}

\begin{IEEEbiography}[{\includegraphics[width=1in,height=1.25in,clip,keepaspectratio]{img/Garcia.jpg}}]{Mario Garc\'{i}a Valdez} 
  Dr. Garc\'{i}a-Valdez is a full-time research professor at the Tijuana
  Institute of Technology. He's interested in the personalization of
  interactive systems, voluntary computing, parallel evolutionary
  computation, interactive evolutionary computation.
\end{IEEEbiography}

\begin{IEEEbiography}[{\includegraphics[width=1in,height=1.25in,clip,keepaspectratio]{img/jj-2016-10.jpg}}]{JJ Merelo Guerv\'{o}s} 
  JJ Merelo is professor at the university of Granada, where he obtained a
  degree in Theoretical Physics and a PhD in Physics in 1994. He's mainly
  interested in evolutionary algorithms, open source software and complex
  systems.
  
\end{IEEEbiography}

\begin{IEEEbiography}[{\includegraphics[width=1in,height=1.25in,clip,keepaspectratio]{img/Puga.jpg}}]{Manuel
    Casta\~{n}\'{o}n Puga} Manuel Casta\~{n}\'{o}n-Puga is Professor at Autonomous University of
  Baja California. He obtained a PhD on Computer Sciences from Autonomous
  University of Baja California in Mexico, and a Masters on Computer Sciences and
  Bachelor in Engineering at Tijuana Technology Institute, in his native Tijuana,
  Mexico. His research of modeling and simulation, agent-base simulation,
  hybrid-intelligent agents and multi-agent systems explores the way in which
  software agents could be used to describe multidimensional environments,
  innovation, evolution and adaptation in complex adaptive systems. Dr.
  Casta\~{n}\'{o}n-Puga collaborates with multidisciplinary researchers and scientists to
  create multidimensional computer simulations of societies, political ideologies,
  trading economies and urban landscapes. His research also intends to incorporate
  the ideas of complexity into the mainstream of engineering and in particular to
  its instruction at the undergraduate level. \end{IEEEbiography}


\begin{IEEEbiography}[{\includegraphics[width=1in,height=1.25in,clip,keepaspectratio]{img/Castillo.png}}]{Oscar
    Castillo L\'{o}pez} Oscar Castillo holds the Doctor in Science degree (Doctor
  Habilitatus) in Computer Science from the Polish Academy of Sciences.
  He is a Professor of Computer Science in the Graduate Division, Tijuana
  Institute of Technology, Tijuana, Mexico. In addition, he is serving as Research
  Director of Computer Science and head of the research group on Hybrid Fuzzy
  Intelligent Systems. Currently, he is President of HAFSA (Hispanic American
  Fuzzy Systems Association) and Past President of IFSA (International Fuzzy
  Systems Association). Prof. Castillo is also Chair of the Mexican Chapter of the
  Computational Intelligence Society (IEEE). He also belongs to the Technical
  Committee on Fuzzy Systems of IEEE and to the Task Force on ``Extensions to
  Type-1 Fuzzy Systems''. He is currently Associate Editor of the Information
  Sciences Journal, Applied Soft Computing Journal, Journal of Engineering
  Applications on Artificial Intelligence, Granular Computing Journal and the
  International Journal of Fuzzy Systems. Finally, he recently received the Recognition as
  Highly Cited Researcher in 2017 and 2018 by Clarivate Analytics and Web of
  Science \end{IEEEbiography}

\bibliography{bibliography}
\bibliographystyle{IEEEtran}

\EOD

\end{document}
